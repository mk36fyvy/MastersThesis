\chapter{Outlook}
    \label{cha:outlook}
    %

    %
    As already stated in the discussion, this work partly serves as a preevaluation in order to provide some insight into the influential factors and their impact on bistability and macrostate switching. As such, it offers information that can be used for educated guesses in a next research step when it comes to systematically quantifying the effects of all the examined parameters and additional ones not analysed in this work.\\
    %

    %
    Other than that, during work on this thesis, some additional topics arose that could be worth addressment in the future.
    %

    %
    \begin{itemize}
        {
            \item \textbf{Macrostate length:} As discussed in \ref{subsec:macrostateLength}, it was not possible to gather sufficient data for reliable statistical analysis. With enough timely and spatial resources, however, this might be a topic worth exploring.
            \item \textbf{Variable dissociation rate:} Allowing considerably different dissociation rates per enzyme type could lead to results which are more suitable from a biological point of view \cite{Tanner2000HATKinetics}. Handling the complexity which arises with this undertaking will be a major challenge, but might be worth the effort.
            \item \textbf{Cooperative enzyme specificity:} As mentionned in the discussion, increasing the rigidity of cooperative enzymes might raise appeal on the biological side.
            \item \textbf{Implement 3D chromatin structure:} Schuettengruber et al. in \cite{schuettengruber2017genome} (p44f) explain that PREs (Polycomb response elements) are responsible for the chromatin forming loops (TADs = topologically associating domains) and are thus able to form large silenced areas of condensed chromatin. These 3D-formations are critical for HOX gene regulation. It is also known that many active gene promoters interact with their enhancers and other promoters in a 3D-fashion \cite{javierre2016lineage}. These findings suggest that it might be beneficial for \ed/ simulations to further explore possibilities to emulate fixed and dynamic 3D interactions within the nucleosome string.
        }
    \end{itemize}
    %
    %
%
%
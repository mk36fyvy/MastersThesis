\chapter{Results}
    \label{cha:results}
    %
    \section{Non-cooperative enzymes do not entail bistable systems}
        %
        Considering a system without cooperative enzymes, that thus only contains enzymes with a context that includes the next neighbours at most, bistability can not be achieved. Indeed, fig. \ref{img:linearCase} discloses a monostable system, as the histogram shows an unimodal distribution throughout the simulation.\\ % Link to more simulation results in appendix?
        %
        \begin{figure}[htpb!]
            \centering
            % \includegraphics[width=0.7\textwidth]{}
            \caption{Boring monostable case. The rule sets used the following enzymes: ? (Rates and additional runs can be found in Appendix.)}
            \label{img:linearCase}
        \end{figure}
        %
        Fig. \ref{img:linearCase} exemplarily shows that the total number of active and silent states ambulate around one same value. In fig. % TODO make figure
        , one can see that even though the overall number of active and silent states changes over time, the overall order in the system stays invariant as the only enzymatic actions are taking place in between the two modification areas at their border. % TODO Discussion: In a case of symmetrical rule set rates, the border will always stay in the middle, because stochastics?
        %
        \begin{itemize}
            {
                \color{red}
                \item Include a boring graph that does not change much. Emphasis on the histogram.  % _,.-^-.,_
                \item Renew the graphs
                    \begin{itemize}
                        \item Gillespie time instead of step number on x-axis
                        \item remove bivalency window where it isn't needed
                        \item print (not plot) the total step number and indicate it in the caption
                        \item remove title
                    \end{itemize}
                \item run the heat map plots on these in order to see if there are to areas or if they are homogeneously mixed
            }
        \end{itemize}
        %
        %
    \section{Bistability on a non-cyclic nucleosome string}
        %
        \begin{itemize}
            {
                \color{red}
                \item Explain bistability and low frequency switching
            }
        \end{itemize}
        %
    %
    %
    \section{Bistability on a cyclic nucleosome string}
        %
        \subsection{Achieving bistability/Probability of bivalency}
            %
            \begin{itemize}
                {
                    \color{red}
                    \item case without switchings
                    \item case with frequent switchings
                }
            \end{itemize}
            %
        %
        %
        \subsection{Influence of dissociation rate on system's noise}
            %
            \begin{itemize}
                {
                    \color{red}
                    \item high vs. low dissociation rate
                }
            \end{itemize}
            %
        %
        %
        \subsection{The boundaries of bistability}
            %
            \begin{itemize}
                {
                    \color{red}
                    \item reduce space (=reach) of the cooperative enzymes
                    \item run without unmodified state
                }
            \end{itemize}
            %
        %
        %
    %
    %
    \section{Bivalency} % TODO Should I include these  cases into the others above? Pro: I don't find anything ground-breaking here. Con: I have an entirely different system which might lead to confusion
        %
        \begin{itemize}
            {
                \color{red}
                \item Here, we are at Kx+Ky
                \item Two systems that either favour bivalency or total active/silent states as an introduction to bivalency
                \item Frequent switching and bivalency
            }
        \end{itemize}
        %
    %
    %
%
%
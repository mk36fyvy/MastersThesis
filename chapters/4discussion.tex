\chapter{Discussion}
    \label{cha:discussion}
    %
    \begin{itemize}
        {
            \color{red}
            \item Summarize the findings.
        }
    \end{itemize}
    %

    %
    \section{Conditions for bistability}
        %
        Hence, even though, there is evidence about the occurrence of fully activated and fully silenced states, this is only a statistical effect. The probability of the system being in an intermediate state is way higher, which disqualifies this system of being an effective "switch" between two extreme states.
        %

        %
        \begin{itemize}
            {
                \color{red}
                \item Bistability (condition: cooperativity) (Explain difference between biological and Sneppen coop.) (Errors in Sneppen's insights on the topic)
                \item Impact of dissociation rate: The dissociation rate was not taken into account for most of the simulations. Setting one and the same high dissociation rate that hardly competes with the association rates of the enzymes smoothens the nucleosome state histograms and thus reduces noise in the system. In terms of epigenetic landscapes, a variable dissociation rate would introduce much complexity into the system, as the number of relevant states would increase. This is the case because every state now would not only be sufficingly described by its modification states but also by the number and type of enzymes bound to the nucleosomes. However, even though the complexity increases significantly, there would not be any inherent biological experience gain from this.
            }
        \end{itemize}
        %
    %
    %
    \section{Proposed mechanism for bistable switching}
        %
        \subsection{Bistable switching: a biologically useful phenomenon}
            %

            %
        %
        %
        \subsection{cyclic}
            %

            %
        %
        %
        \subsection{non-cyclic}
            %
            \begin{itemize}
                {
                    \color{red}
                    \item A state seed will immediately be extinguished by an opposing cooperative remover. In order to prevent this, there are two possibilities:
                        \begin{itemize}
                            \item Build the seed in a corner where the coop Remover cannot be active
                            \item Remove coop Removers
                        \end{itemize}
                }
            \end{itemize}
            %
        %
        %
    %
    %
%
%
